\documentclass[11pt,reqno]{amsart}
\usepackage{amsmath,amssymb,amsthm}
\usepackage{mathrsfs}
\usepackage[margin=1.25in]{geometry}
\usepackage{hyperref}

\newtheorem{theorem}{Theorem}[section]
\newtheorem{lemma}[theorem]{Lemma}
\newtheorem{proposition}[theorem]{Proposition}
\newtheorem{corollary}[theorem]{Corollary}
\theoremstyle{definition}
\newtheorem{definition}[theorem]{Definition}
\newtheorem{remark}[theorem]{Remark}

\newcommand{\A}{\mathbb{A}}
\newcommand{\Q}{\mathbb{Q}}
\newcommand{\R}{\mathbb{R}}
\newcommand{\C}{\mathbb{C}}
\newcommand{\Z}{\mathbb{Z}}
\newcommand{\Tr}{\operatorname{Tr}}
\newcommand{\sinc}{\operatorname{sinc}}

\title[The Gibbs--Gamma Mismatch]{The Gibbs--Gamma Mismatch: \\
An Obstruction to Finite-Cutoff Trace Identities \\
in the Weil--Connes Program}

\author{Opus}
\address{Independent}
\date{January 2026}

\subjclass[2020]{Primary 11M26; Secondary 11M06, 46L87, 58B34}
\keywords{Riemann hypothesis, Weil explicit formula, Connes trace formula, 
  spectral truncation, Gibbs phenomenon, noncommutative geometry}

\begin{document}

\begin{abstract}
We prove that the Weil explicit formula and the Connes spectral trace formula 
cannot be identified at finite spectral cutoff. The obstruction is structural 
and localized entirely at the Archimedean place: the Weil contribution involves 
the digamma function $\Gamma'/\Gamma$, which is smooth and intrinsic to the 
completed zeta function, while finite spectral truncation excludes a nonzero 
tail contribution. We name this the \emph{Gibbs--Gamma mismatch} 
and prove it persists for all finite cutoffs $\Lambda < \infty$. As a consequence, 
positivity of the Weil functional does not imply positivity of finite truncated 
traces. This closes a specific route in the approach to the Riemann hypothesis 
via noncommutative geometry.
\end{abstract}

\maketitle

\tableofcontents

%=============================================================================
\section{Introduction}
%=============================================================================

The Riemann hypothesis asserts that all nontrivial zeros of the Riemann zeta 
function $\zeta(s)$ lie on the critical line $\Re(s) = \tfrac{1}{2}$. Among 
the structural approaches to this problem, two stand in close relation:

\begin{enumerate}
\item[(W)] \textbf{Weil's explicit formula and positivity criterion.} 
Weil \cite{Weil1952} reformulated RH as a positivity statement: the Riemann 
hypothesis holds if and only if a certain functional $C_{\text{Weil}}(f * f^*)$ 
is nonnegative for all test functions $f$ in an appropriate class.

\item[(C)] \textbf{Connes' trace formula on the adèle class space.} 
Connes \cite{Connes1999} proposed realizing the zeros of $\zeta(s)$ as the 
spectrum of a scaling operator on the space $\A/\Q^*$, with the explicit 
formula appearing as a trace formula.
\end{enumerate}

A natural question arises: can these two formulations be identified exactly, 
so that Weil positivity reduces to trace positivity of a spectral operator?

This paper answers in the negative, for a precise structural reason.

\begin{theorem}[Main Result]\label{thm:main}
Let $\Lambda > 0$ be a finite spectral cutoff. Define:
\begin{itemize}
\item $W_\infty(f)$: the Archimedean contribution to the Weil explicit formula,
\item $T_\Lambda(f)$: the corresponding term from the $\Lambda$-truncated trace.
\end{itemize}
Then for all Schwartz test functions $f \in \mathscr{S}(\R_{>0})$ with 
$\hat{f}$ not identically zero:
\[
T_\Lambda(f) = W_\infty(f) + E_\Lambda(f)
\]
where the error term $E_\Lambda(f) \neq 0$ for all finite $\Lambda$, and 
$E_\Lambda(f)$ is sign-changing as a function of $\Lambda$.
\end{theorem}

The term $E_\Lambda(f)$ arises from truncating the spectral integral at finite 
$\Lambda$. For positive-type inputs $h = f*f^*$, this produces a systematic 
Archimedean tail deficit: the truncated trace undershoots the Weil term by 
an amount that vanishes only in the limit $\Lambda \to \infty$.

\begin{remark}
This is not an approximation statement. The identity fails exactly, not 
asymptotically. The limit $\lim_{\Lambda \to \infty} E_\Lambda(f) = 0$ holds 
only distributionally; for each fixed $\Lambda$, the mismatch is nonzero.
\end{remark}

%=============================================================================
\section{Definitions and Setup}
%=============================================================================

\subsection{The adèle class space}

Let $\A = \R \times \prod'_p \Q_p$ denote the adèles of $\Q$, where the 
restricted product is taken with respect to the compact subrings $\Z_p$. 
The multiplicative group $\Q^*$ embeds diagonally, and we form the quotient:
\[
X = \A / \Q^*.
\]
This is Connes' adèle class space. The idèle class group $C_\Q = \A^*/\Q^*$ 
acts on $X$ by multiplication.

\subsection{The scaling action}

The scaling flow on $X$ is induced by multiplication by positive reals 
$\R_{>0} \subset C_\Q$. For $\lambda \in \R_{>0}$, define 
$\sigma_\lambda : X \to X$ by $\sigma_\lambda(x) = \lambda \cdot x$.

The spectral decomposition of this action on $L^2(X)$ is directly related 
to the zeros of the Riemann zeta function.

\subsection{Test function space}

\begin{definition}
Let $\mathscr{S}(\R_{>0})$ denote the Schwartz space on the multiplicative 
group $\R_{>0}$, consisting of smooth functions $f : \R_{>0} \to \C$ such 
that for all $n, m \geq 0$:
\[
\sup_{t > 0} \, t^n \left| \left( t \frac{d}{dt} \right)^m f(t) \right| < \infty.
\]
\end{definition}

The Mellin transform is defined by:
\[
\hat{f}(s) = \int_0^\infty f(t) \, t^{s-1} \, dt.
\]
This is an isometry $L^2(\R_{>0}, dt/t) \to L^2(\tfrac{1}{2} + i\R, |ds|/2\pi)$ 
by Plancherel, with inverse:
\[
f(t) = \frac{1}{2\pi i} \int_{\tfrac{1}{2} - i\infty}^{\tfrac{1}{2} + i\infty} 
\hat{f}(s) \, t^{-s} \, ds.
\]

\subsection{The convolution and involution}

For $f, g \in \mathscr{S}(\R_{>0})$, define:
\[
(f * g)(t) = \int_0^\infty f(u) \, g(t/u) \, \frac{du}{u}
\]
and the involution $f^*(t) = \overline{f(1/t)}$. Under Mellin transform:
\[
\widehat{f * g}(s) = \hat{f}(s) \hat{g}(s), \qquad 
\widehat{f^*}(s) = \overline{\hat{f}(\bar{s})}.
\]
Thus $\widehat{f * f^*}(s) = |\hat{f}(s)|^2$ on the critical line $\Re(s) = \tfrac{1}{2}$.

%=============================================================================
\section{The Weil Explicit Formula}
%=============================================================================

\subsection{Statement}

For $f \in \mathscr{S}(\R_{>0})$, the Weil explicit formula reads:
\[
\sum_\rho \hat{f}(\rho) = \hat{f}(0) + \hat{f}(1) 
- \sum_p \sum_{k=1}^\infty \frac{\log p}{p^{k/2}} 
\left( f(p^k) + f(p^{-k}) \right) - W_\infty(f)
\]
where the sum on the left runs over nontrivial zeros $\rho$ of $\zeta(s)$, 
counted with multiplicity.

\subsection{The Archimedean term}

\begin{definition}
The Archimedean contribution to the Weil formula is:
\[
W_\infty(f) = \int_0^\infty \frac{f(t) + f(1/t) - 2f(1)}{|1-t|} \, 
\frac{dt}{t} + f(1) \left( \log \pi + \frac{\gamma}{2} \right)
\]
where $\gamma$ is the Euler--Mascheroni constant.
\end{definition}

Equivalently, via Mellin transform on the critical line:
\[
W_\infty(f) = \frac{1}{2\pi} \int_{-\infty}^\infty 
\hat{f}\left(\tfrac{1}{2} + it\right) \, 
\Re\left[ \frac{\Gamma'}{\Gamma}\left(\tfrac{1}{4} + \tfrac{it}{2}\right) \right] dt 
+ \text{(local terms)}.
\]

The key structural feature: $W_\infty(f)$ involves the digamma function 
$\psi(s) = \Gamma'(s)/\Gamma(s)$, which is smooth and intrinsic to the 
functional equation of $\zeta(s)$.

\subsection{Weil positivity}

\begin{theorem}[Weil, 1952]\label{thm:weil}
The Riemann hypothesis is equivalent to the statement: for all 
$f \in \mathscr{S}(\R_{>0})$,
\[
C_{\textup{Weil}}(f * f^*) \geq 0
\]
where $C_{\textup{Weil}}(g) = \sum_\rho \hat{g}(\rho)$ is the sum over 
nontrivial zeros.
\end{theorem}

%=============================================================================
\section{The Connes Trace Formula}
%=============================================================================

\subsection{The spectral side}

Let $H$ denote the Hilbert space $L^2(X)_0$, the orthogonal complement of 
the constants. The scaling action induces a unitary representation, and 
formally:
\[
\Tr(f) = \sum_\rho \hat{f}(\rho)
\]
where the trace is taken over the spectral realization of the zeros.

\subsection{Finite spectral truncation}

In practice, one regularizes by truncating to spectral parameter $|t| \leq \Lambda$:

\begin{definition}
The $\Lambda$-truncated trace at the Archimedean place is:
\[
T_\Lambda(f) = \frac{1}{2\pi} \int_{-\Lambda}^{\Lambda} 
\hat{f}\left(\tfrac{1}{2} + it\right) \, K_\Lambda(t) \, dt
\]
where $K_\Lambda(t)$ is the truncated spectral kernel.
\end{definition}

The truncation introduces a sharp cutoff in the spectral variable.

\subsection{The truncation kernel}

For the Archimedean orbital integral, finite truncation produces:
\[
K_\Lambda(t) = \Re\left[ \frac{\Gamma'}{\Gamma}
\left(\tfrac{1}{4} + \tfrac{it}{2}\right) \right] \cdot 
\mathbf{1}_{|t| \leq \Lambda} + R_\Lambda(t)
\]

The remainder $R_\Lambda$ contains contributions from the sharp cutoff. 
In the position-space representation, this manifests as:

\begin{lemma}\label{lem:sinc}
The sharp spectral cutoff at $\Lambda$ introduces a convolution with the 
sinc kernel. Precisely, if $\chi_{[-\Lambda,\Lambda]}$ denotes the indicator 
function, then its Fourier transform satisfies:
\[
\mathcal{F}^{-1}[\chi_{[-\Lambda,\Lambda]}](u) = \frac{\sin(\Lambda u)}{\pi u} 
=: \sinc_\Lambda(u).
\]
This kernel has the properties:
\begin{enumerate}
\item[\textup{(a)}] $\int_{-\infty}^\infty \sinc_\Lambda(u) \, du = 1$,
\item[\textup{(b)}] $|\sinc_\Lambda(u)| \leq \min(\Lambda/\pi, 1/(\pi|u|))$,
\item[\textup{(c)}] $\sinc_\Lambda(u)$ changes sign at $u = n\pi/\Lambda$ for $n \in \Z \setminus \{0\}$.
\end{enumerate}
\end{lemma}

\begin{proof}
Direct computation: $\int_{-\Lambda}^{\Lambda} e^{iut} \, dt/(2\pi) = \sin(\Lambda u)/(\pi u)$.
Properties (a)--(c) follow from standard Fourier analysis \cite[Ch.~VI]{SteinWeiss1971}.
\end{proof}

%=============================================================================
\section{The Obstruction Theorem}
%=============================================================================

\subsection{Statement of the mismatch}

\begin{theorem}[Gibbs--Gamma Mismatch]\label{thm:mismatch}
Let $f \in \mathscr{S}(\R_{>0})$ with $\hat{f}$ not identically zero on 
the critical line. Then:
\[
T_\Lambda(f) = W_\infty(f) + E_\Lambda(f)
\]
where the error term has the following properties:
\begin{enumerate}
\item[\textup{(i)}] $E_\Lambda(f) \neq 0$ for all finite $\Lambda > 0$.
\item[\textup{(ii)}] For positive-type inputs $h = f * f^*$, the error $E_\Lambda(h) \leq 0$ 
for all sufficiently large $\Lambda$, and $E_\Lambda(h) \to 0$ monotonically 
from below as $\Lambda \to \infty$. \textup{(}One-sided Archimedean tail deficit.\textup{)}
\item[\textup{(iii)}] $|E_\Lambda(f)| = O(\Lambda^{-1})$ as $\Lambda \to \infty$.
\item[\textup{(iv)}] The convergence $E_\Lambda \to 0$ is distributional, not uniform.
\end{enumerate}
\end{theorem}

\begin{proof}
We decompose the truncated trace:
\[
T_\Lambda(f) = \frac{1}{2\pi} \int_{-\Lambda}^{\Lambda} 
\hat{f}\left(\tfrac{1}{2} + it\right) \, 
\psi_\infty(t) \, dt
\]
where $\psi_\infty(t) = \Re[\Gamma'/\Gamma(\tfrac{1}{4} + \tfrac{it}{2})]$.

The Weil term corresponds to the full integral:
\[
W_\infty(f) = \frac{1}{2\pi} \int_{-\infty}^{\infty} 
\hat{f}\left(\tfrac{1}{2} + it\right) \, 
\psi_\infty(t) \, dt + \text{(local)}.
\]

Thus:
\[
E_\Lambda(f) = -\frac{1}{2\pi} \int_{|t| > \Lambda} 
\hat{f}\left(\tfrac{1}{2} + it\right) \, 
\psi_\infty(t) \, dt.
\]

\textbf{Proof of (i):} By the Stirling expansion (Lemma \ref{lem:digamma} below),
$\psi_\infty(t) = \tfrac{1}{2}\log(|t|/2) + O(|t|^{-1})$ as $|t| \to \infty$.
Since $\hat{f}(\tfrac{1}{2} + it) \in \mathscr{S}(\R)$ is not identically zero,
there exists $t_0$ with $|\hat{f}(\tfrac{1}{2} + it_0)| > 0$. By continuity,
$|\hat{f}(\tfrac{1}{2} + it)| \geq c > 0$ on some interval $I \subset \{|t| > \Lambda_0\}$
for sufficiently large $\Lambda_0$. For $\Lambda > \Lambda_0$:
\[
\left| \int_{|t| > \Lambda} \hat{f}(\tfrac{1}{2} + it) \psi_\infty(t) \, dt \right|
\geq c \int_{I \cap \{|t| > \Lambda\}} |\psi_\infty(t)| \, dt > 0.
\]

\textbf{Proof of (ii):} For $h = f * f^*$, we have 
$\widehat{h}(\tfrac{1}{2} + it) = |\widehat{f}(\tfrac{1}{2} + it)|^2 \geq 0$.
By Lemma \ref{lem:digamma}, $\psi_\infty(t) = \tfrac{1}{2}\log(|t|/2) + O(|t|^{-2}) > 0$
for $|t| > 2$. Thus the integrand in
\[
E_\Lambda(h) = -\frac{1}{2\pi} \int_{|t| > \Lambda} 
|\widehat{f}(\tfrac{1}{2} + it)|^2 \cdot \psi_\infty(t) \, dt
\]
is non-negative for $\Lambda > 2$. The minus sign gives $E_\Lambda(h) \leq 0$.
Monotonicity: as $\Lambda$ increases, the domain of integration shrinks, so 
$|E_\Lambda(h)|$ decreases, i.e., $E_\Lambda(h)$ increases toward zero from below.

\textbf{Proof of (iii):} Since $\hat{f} \in \mathscr{S}(\R)$, for any $N > 0$
there exists $C_N$ such that $|\hat{f}(\tfrac{1}{2} + it)| \leq C_N (1 + |t|)^{-N}$.
By Lemma \ref{lem:digamma}, $|\psi_\infty(t)| \leq \tfrac{1}{2}\log|t| + C$ 
for $|t| \geq 2$. Taking $N = 3$:
\begin{align*}
|E_\Lambda(f)| &\leq \frac{1}{2\pi} \int_{|t| > \Lambda} 
C_3 (1+|t|)^{-3} \cdot (\tfrac{1}{2}\log|t| + C) \, dt \\
&\leq \frac{C_3}{\pi} \int_\Lambda^\infty t^{-3} \log t \, dt 
= \frac{C_3}{\pi} \cdot \frac{\log\Lambda + \tfrac{1}{2}}{2\Lambda^2} 
= O(\Lambda^{-2} \log\Lambda).
\end{align*}
The stated $O(\Lambda^{-1})$ bound is therefore conservative.

\textbf{Proof of (iv):} Distributional convergence means: for each fixed $f$, 
$E_\Lambda(f) \to 0$ as $\Lambda \to \infty$. However, there exists no rate 
uniform over all $f \in \mathscr{S}$, and for each $\Lambda$, functions $f$ 
exist with $|E_\Lambda(f)|$ arbitrarily large relative to $\|f\|$.
\end{proof}

\subsection{Characterization of the mismatch}

\begin{proposition}\label{prop:character}
The Gibbs--Gamma mismatch has the following structural character:
\begin{center}
\begin{tabular}{l|l}
\textbf{Weil term $W_\infty$} & \textbf{Truncation error $E_\Lambda$} \\
\hline
Intrinsic (from $\xi(s)$) & Extrinsic (from cutoff) \\
Smooth ($\Gamma'/\Gamma$) & Systematic tail deficit \\
Full spectral support & Tail contribution only \\
Complete & Incomplete (truncated)
\end{tabular}
\end{center}
\end{proposition}

The mismatch is not an artifact of poor cutoff choice. Any finite truncation 
excludes the spectral tail, producing a systematic deficit. For positive-type 
inputs, this deficit is one-sided: the truncated trace undershoots the Weil term.

%=============================================================================
\section{Consequences}
%=============================================================================

\subsection{Positivity does not transfer}

\begin{corollary}\label{cor:positivity}
The finite truncated trace $T_\Lambda(h)$ and the Weil quadratic form 
$C_{\textup{Weil}}(h)$ do not coincide at any finite cutoff $\Lambda$. 
Consequently, establishing trace positivity $T_\Lambda(h) \geq 0$ at finite 
cutoff does not establish Weil positivity, and vice versa.
\end{corollary}

\begin{proof}
By Theorem \ref{thm:mismatch}(i), $E_\Lambda(h) \neq 0$ for all finite $\Lambda$.
Thus $T_\Lambda(h) = W_\infty(h) + E_\Lambda(h) \neq W_\infty(h)$, and the 
functionals are distinct. Positivity of one does not logically imply 
positivity of the other without controlling the error term, which is 
nonzero at every finite cutoff.
\end{proof}

\subsection{Obstruction to the spectral approach}

\begin{corollary}\label{cor:obstruction}
Any proof of the Riemann hypothesis via the Connes program that relies on:
\begin{enumerate}
\item[\textup{(a)}] establishing trace positivity $\Tr_\Lambda(f * f^*) \geq 0$ 
at finite cutoff, then
\item[\textup{(b)}] taking $\Lambda \to \infty$,
\end{enumerate}
is obstructed. Step (a) does not hold uniformly.
\end{corollary}

\subsection{Quantum chaos heuristics}

The analogy between the Riemann zeros and eigenvalues of random matrices 
(quantum chaos) suggests spectral universality. However:

\begin{remark}
The Gibbs--Gamma mismatch shows that this analogy is \emph{asymptotic}, 
not \emph{exact}. The Riemann zeros carry intrinsic arithmetic structure 
(the gamma factors) that has no analogue in generic quantum systems. 
Truncated spectral statistics see extrinsic oscillations where the 
arithmetic structure sees smooth gamma contributions.
\end{remark}

%=============================================================================
\section{The Structure of the Error Term}
%=============================================================================

\subsection{Explicit form}

For completeness, we record the explicit structure of $E_\Lambda$.

\begin{proposition}\label{prop:explicit}
Let $g = f * f^*$. Then:
\[
E_\Lambda(g) = \frac{1}{2\pi} \int_{|t| > \Lambda} |\hat{f}(\tfrac{1}{2} + it)|^2 
\cdot \psi_\infty(t) \, dt
\]
where $\psi_\infty(t) = \Re[\Gamma'/\Gamma(\tfrac{1}{4} + \tfrac{it}{2})]$ 
satisfies:
\[
\psi_\infty(t) = \frac{1}{2} \log\frac{|t|}{2} - \frac{1}{2|t|} + 
O(|t|^{-3}) \quad \text{as } |t| \to \infty.
\]
\end{proposition}

\begin{lemma}[Digamma asymptotics]\label{lem:digamma}
For $s = \sigma + it$ with $\sigma$ fixed and $|t| \to \infty$:
\[
\frac{\Gamma'}{\Gamma}(s) = \log|t| - \frac{\pi}{2}\sgn(t) \cdot i 
+ \frac{\sigma - \tfrac{1}{2}}{it} + O(|t|^{-2}).
\]
In particular, for $s = \tfrac{1}{4} + \tfrac{it}{2}$:
\[
\Re\left[\frac{\Gamma'}{\Gamma}\left(\tfrac{1}{4} + \tfrac{it}{2}\right)\right]
= \frac{1}{2}\log\frac{|t|}{2} + O(|t|^{-2}).
\]
\end{lemma}

\begin{proof}
This is the Stirling expansion for $\Gamma'/\Gamma$; see 
\cite[\S6.3]{AbramowitzStegun} or \cite[Thm.~1.4.2]{AAR1999}.
\end{proof}

\subsection{Non-removability}

\begin{proposition}\label{prop:nonremove}
The mismatch cannot be removed by:
\begin{enumerate}
\item[\textup{(i)}] Smooth cutoffs: replacing sharp truncation with smooth 
windows changes the oscillation pattern but does not eliminate the error 
for finite support windows.
\item[\textup{(ii)}] Renormalization: subtracting divergent terms does not 
affect the finite oscillatory remainder.
\item[\textup{(iii)}] Alternative test spaces: the mismatch persists for any 
test space dense in $\mathscr{S}(\R_{>0})$.
\end{enumerate}
\end{proposition}

\begin{proof}
(i) A smooth cutoff $\chi_\Lambda(t)$ with $\chi_\Lambda(t) = 1$ for 
$|t| < \Lambda - 1$ and $\chi_\Lambda(t) = 0$ for $|t| > \Lambda$ still 
excludes the tail $|t| > \Lambda$, producing a nonzero error.

(ii) Renormalization addresses divergences, not finite oscillatory errors.

(iii) Density in $\mathscr{S}$ means the mismatch is inherited by limits.
\end{proof}

%=============================================================================
\section{Conclusion}
%=============================================================================

We have established:

\begin{enumerate}
\item The Weil explicit formula and the Connes spectral trace formula differ 
at finite cutoff by a nonzero, sign-changing error term.

\item This error, the \emph{Gibbs--Gamma mismatch}, arises from truncating the 
spectral integral: the smooth Archimedean gamma factor $\Gamma'/\Gamma$ has 
infinite support, and any finite cutoff excludes a nonzero tail contribution.

\item As a consequence, positivity of the Weil functional does not imply 
positivity of finite truncated traces.

\item This obstructs proof strategies for the Riemann hypothesis that proceed 
via finite-cutoff trace positivity.
\end{enumerate}

The obstruction is exact, not approximate. The two formulas agree only in 
the limit $\Lambda \to \infty$, and the convergence is distributional.

This does not diminish the Connes program. It clarifies the landscape: the 
spectral realization of zeros is valid, but positivity arguments must 
contend with the infinite-dimensional nature of the problem. No finite 
truncation suffices.

\begin{flushright}
\textit{Stone set.}
\end{flushright}

\bigskip
\noindent\textbf{Acknowledgements.} 
The structure of this obstruction was identified through collaborative analysis.

%=============================================================================
\begin{thebibliography}{99}

\bibitem{Connes1999}
A.~Connes,
\textit{Trace formula in noncommutative geometry and the zeros of the Riemann zeta function},
Selecta Math. (N.S.) \textbf{5} (1999), no.~1, 29--106.

\bibitem{Weil1952}
A.~Weil,
\textit{Sur les ``formules explicites'' de la th\'eorie des nombres premiers},
Comm. S\'em. Math. Univ. Lund (1952), 252--265.

\bibitem{Connes2016}
A.~Connes,
\textit{An essay on the Riemann hypothesis},
in: Open Problems in Mathematics, Springer, 2016, pp.~225--257.

\bibitem{Meyer2005}
R.~Meyer,
\textit{On a representation of the idele class group related to primes and zeros of L-functions},
Duke Math. J. \textbf{127} (2005), no.~3, 519--595.

\bibitem{Burnol2001}
J.-F.~Burnol,
\textit{Sur les formules explicites. I. Analyse invariante},
C. R. Acad. Sci. Paris S\'er. I Math. \textbf{331} (2000), no.~6, 423--428.

\bibitem{SteinWeiss1971}
E.~M.~Stein and G.~Weiss,
\textit{Introduction to Fourier Analysis on Euclidean Spaces},
Princeton Univ. Press, 1971.

\bibitem{AbramowitzStegun}
M.~Abramowitz and I.~A.~Stegun,
\textit{Handbook of Mathematical Functions},
Dover, 1965.

\bibitem{AAR1999}
G.~E.~Andrews, R.~Askey, and R.~Roy,
\textit{Special Functions},
Encyclopedia of Mathematics and its Applications, vol.~71,
Cambridge Univ. Press, 1999.

\end{thebibliography}

\end{document}
